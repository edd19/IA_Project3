\documentclass[10pt,a4paper]{article}
\usepackage[utf8]{inputenc}
\usepackage{amsmath}
\usepackage{amsfonts}
\usepackage{amssymb}
\usepackage[]{algorithm2e}
\usepackage{graphicx}
\usepackage[left=2cm,right=2cm,top=2cm,bottom=2cm]{geometry}

\begin{document}

\title{LINGI2261: Artificial Intelligence \\
Assignement 3 : Adversarial Search}
\author{Group 8: Ndizera Eddy \and El Jilali Solaiman}
\date{\today}
\maketitle

\section{Alpha-Beta search}

\subsection{Perform the MiniMax algorithm on the tree in Figure 1, i.e. put a value to each node. Circle the move the root player should do.}

\begin{figure}[h]
\includegraphics[scale=0.4]{img/minimax.jpg} 
\caption{\label{minimax} MiniMax}
\end{figure}

\subsection{Perform the Alpha-Beta algorithm on the tree in Figure 2. At each non terminal node, put the successive values of $ \alpha $ and $ \beta $. Cross out the arcs reaching non visited nodes. Assume a left-to-right node expansion.}

\begin{figure}[h]
\includegraphics[scale=0.4]{img/alphabeta_left.jpg} 
\caption{\label{alphabetaleft} Alpha-Beta, left-to-right expansion}
\end{figure}

\subsection{Do the same, assuming a right-to-left node expansion instead (Figure 3).}

\begin{figure}[h]
\includegraphics[scale=0.4]{img/alphabeta_right.jpg} 
\caption{\label{alphabetaright} Alpha-Beta, right-to-left expansion}
\end{figure}

\subsection{Can the nodes be ordered in such a way that Alpha-Beta pruning can cut off more branches (in a left-to right node expansion)? If no, explain why; if yes, give the new ordering and the resulting new pruning.}

Yes, if we order the nodes in increasing order like shown in Figure \ref{alphabetanodes}, the Alpha-Beta pruning will cut off more branches. The nodes must be set in that order because it's MIN that chooses the action to pick in the bottom of the tree.

\begin{figure}[h]
\includegraphics[scale=0.4]{img/alphabeta_nodes.jpg} 
\caption{\label{alphabetanodes} Alpha-Beta, left-to-right expansion}
\end{figure}

\section{Avalam}

\subsection{A basic Alpha-Beta Agent}

The basic Alpha-Beta Agent can be found on INGInious.

\subsection{Comparison of two evalutaion functions}

\end{document}